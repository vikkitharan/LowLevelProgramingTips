\chapter{C++}
\href{https://stackoverflow.com/questions/388242/the-definitive-c-book-guide-and-list}{The Definitive C++ Book Guide and List}

\section{compile code} 
% using g++/clang

\subsection{makefile} 

\section{Libraries} 
%locations
%LD_LIBRARY
%ldd command

\section{Coding standard} 
\subsection{clang-format}
\subsection{vim integration}
\subsection{automation - git hooks}


\section{Static Error Check} 
\subsection{clang-tidy}
\subsection{vim integration}
\subsection{automation - git hooks}

\section{Move Semantics} 

\subsection{Value Categories} 

The example code is \texttt{./c++/source/moveSematics/lrValues.cpp}.

\subsubsection{L-value} 

\begin{itemize}
    \item has a name
    \begin{itemize}
      \item all variables
    \end{itemize}
    \item can be assigned values
    \item some expressions return l-value
    \begin{itemize}
      \item persists beyond the expression
    \end{itemize}
    \item some functions return l-value
    \item l-value reference - reference to l-value
\end{itemize}

\subsubsection{R-value} 

\begin{itemize}
    \item does not have a name
    \item a temporary value
    \item some expressions return r-value
    \begin{itemize}
      \item does not persist beyond the expression
    \end{itemize}
    \item some functions return r-value
    \item r-value reference - reference to r-value (introduced in C++ 11)
\end{itemize}

A function overloading can be done with l-value and r-value

\subsubsection{R-value References} 

\begin{itemize}
    \item Introduced for move semantics
    \item A reference created to a temporary.
    \begin{itemize}
      \item Represents a temporary
    \end{itemize}
    \item Created with \texttt{\&\&}
\end{itemize}

\subsection{ Copy \& Move Semantics} 
\begin{itemize}
    \item copy constructor create a copy of an object
      \begin{itemize}
        \item Not useful for a temporary objects, for example return from a function
      \end{itemize}
    \item Move constructor move the source object
    \item Created with \texttt{\&\&}
\end{itemize}

\subsection{Resource Management} 
Constructor may acquire resources like pointer, file handler, socket, etc. 
Subsequently, object copy, move and destruction must decide the resource management.
For example, in the destruction the resources must be released. Coping and moving objects must handle the resources accordingly.

\subsection{Rule of 5} 
If a class has ownership semantics (a constructor acquire resources), then you must provide a user-defined and ensure proper handling of the underlying resources.
\begin{itemize}
  \item Destructor -- release the resources.
  \item Copy constructor -- copies the resources.
  \item Copy assignment operator -- copies the resources.
  \item Move constructor -- moves the resources.
  \item Move assignment operator -- moves the resources.
\end{itemize}

\subsection{Rule of 0}
If a class does not has ownership semantics, then you should not provide a user-defined functions. The compiler will synthesize the ncecessary fuctions
% TODO [vikki @22/09/13]: under stand https://en.cppreference.com/w/cpp/language/rule_of_three %
