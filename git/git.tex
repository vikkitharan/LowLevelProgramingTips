\chapter{Git}

% TODO [vikki @22/08/22]: refer my google document slide. %
\section{Configuration}
git has three level of configurations.
\begin{enumerate}
  \item \textit{local}: Local configs are only available for the current repo and stored in \texttt{.git/config} directory.
  \item \textit{global}: Global configs are available for all repos for the current user and stored in \texttt{~/.gitconfig}.
  \item \textit{system}: System configs are available for all the users and repos and stored in \texttt{/etc/gitconfig}.
\end{enumerate}

Running \texttt{git config comit.template ./.git/gitmessage.txt} from a git repo sets commit template message for the repo and it overwrites \texttt{global} and \texttt{system} user name.


Running \texttt{git config --global comit.template ~/.dotfiles/gitmessage.txt} sets the commit template name for all repos and it overwrites \texttt{system} user name.


Running \texttt{git config --system core.editor vim} sets the editor for all repos and users.

\section{short cuts}

\subsection{commit}
Each commits is a story point -- telling the history of the change.

A commit should have a single logical concept.

A commit can be edit, modified or clan up by \texttt{rebase}~\ref{git_rebase} command.

\subsection{tag}
\href{https://git-scm.com/book/en/v2/Git-Basics-Tagging}{git-scm.com: Tagging}
Git has the ability to tag specific points in a repository’s history as being important. Typically, people use this functionality to mark release points (v1.0, v2.0 and so on).


\subsubsection{Creating Tags}%
\label{ssub:subsubsection_name}


 Git supports two types of tags: lightweight and annotated.

 A lightweight tag is very much like a branch that doesn`t change -- it`s just a pointer to a specific commit.

 To create a lightweight tag, \texttt{git tag v1.0.0}.


 Annotated tags, however, are stored as full objects in the Git database. They’re checksummed; contain the tagger name, email, and date; have a tagging message; and can be signed and verified with GNU Privacy Guard (GPG). It’s generally recommended that you create annotated tags so you can have all this information; but if you want a temporary tag or for some reason don’t want to keep the other information, lightweight tags are available too.

 To create a annotated tag, \texttt{git tag a  v1.0.0 -m "This is the alpha release"}.

\subsection{Edit a commit}
When I want to edit the last commit, I stage all necessary changes then I use \texttt{ git commit --amend --no-edit} command.


\subsection{rebase} \label{git_rebase}
Once a branch is ready to merge, I clean up the commits -- history of the changes. 


\texttt{git checkout develop}, checkouts to \texttt{develop} branch.

\texttt{git rebase -i master}, start the clean up commits after \texttt{master} branch.

I check each commits, if there is more than single logical change, I split them.

If I have multiple commits related to a single logical change, I rename the title of those commits.

Then I rearrange the commits such that commits related to a logical change are group together.

Finally I Combine multiple commits into a single commit.


\section{Template Directory}%
\label{sec:template_directory}

\texttt{\$ git init} and \texttt{\$ git clone} copy files and directories whose name do not start with a ``.'' (dot) in the template directory into newly created \texttt{.git} directory, for example \texttt{.git/hooks}.

The default template directory is \texttt{/usr/share/git-core/templates}.

But user defined template directory can be set by one of the following (in order):
\begin{itemize}
  \item the argument given with the \texttt{\$ git init -{}-template absolute/path/of/template} or \texttt{\$ git clone -{}-template absolute/path/of/template}. NOTE path must be absolute ;
  \item the contents of the \texttt{\$GIT\_TEMPLATE\_DIR} environment variable;
  \item the \texttt{init.templateDir} configuration variable; or
\end{itemize}


As dot files are not copied to new git repo, the template directory its self can be version controlled, \texttt{.git} directory or its contents will not be copied.


\subsection{hooks}
\subsubsection{Getting Help}%
\begin{itemize}
  \item \texttt{\$ man githooks}
\end{itemize}

\subsubsection{My git hooks}%

\markdownInput{git/source/README.md}

% TODO [vikki @22/09/20]: update git hooks to fix whitespace error. %
%https://stackoverflow.com/questions/591923/make-git-automatically-remove-trailing-white-space-before-committing
%https://snipplr.com/view/28523/git-precommit-hook-to-fix-trailing-whitespace
%https://makandracards.com/makandra/11541-how-to-not-leave-trailing-whitespace-using-your-editor-or-git

\subsubsection{Resources}%
\begin{itemize}
  \item \href{https://git-scm.com/book/en/v2/Customizing-Git-Git-Hooks}{Git Pro book, 8.3 Customizing Git - Git Hooks} 
  \item \href{https://www.atlassian.com/git/tutorials/git-hooks}{Atlassion: Git Hooks} 
  \item \href{https://githooks.com/}{githooks.com} 
  \item \href{https://www.redhat.com/sysadmin/git-hooks}{RedHat Git hooks: How to automate actions in your Git repo}
  \item \href{https://medium.com/@f3igao/get-started-with-git-hooks-5a489725c639}{Medium: Get stated with Git Hooks} 
\end{itemize}
Git automatically triggers custom scripts (saved in the \texttt{.git/hooks} subdirectory) before (pre-) or after (post-) certain version control events such as: commit, push and merge.

Some of the git hooks' benefits are maintaining style guide standards, encouraging a commit policy, automating development workflow, and implementing continuous integration.

These hook scripts are only limited by a developer's imagination. Some example hook scripts include:
\begin{itemize}
  \item    pre-commit: Check the commit message for spelling errors.
  \item    pre-receive: Enforce project coding standards.
  \item    post-commit: Email/SMS team members of a new commit.
  \item    post-receive: Push the code to production.
\end{itemize}


There are two groups of these hooks: 

\begin{enumerate}
  \item client-side and server-side. Client-side hooks are triggered by operations such as committing and merging
  \item server-side hooks run on network operations such as receiving pushed commits. You can use these hooks for all sorts of reasons.
\end{enumerate}




Every git repository has a \texttt{.git/hooks} subdirectory with some example scripts, many of which are useful by themselves; but they also document the input values of each script. All the examples are written as shell scripts, with some Perl thrown in, but any properly named executable scripts will work fine – you can write them in Ruby or Python or whatever language you are familiar with. If you want to use the bundled hook scripts, you’ll have to rename them; their file names all end with .sample.

To add another hook script, put a file in the \texttt{.git/hooks} subdirectory. It is named without any extension and set executable.

\subsubsection{Client-Side Hooks}%


\textbf{It’s important to note that client-side hooks are local and untracked -- not copied when you clone a repository. You can add the githooks in \href{https://git-scm.com/docs/git-init\#\_template\_directory}{template directory }.}


There are a lot of client-side hooks. This section splits them into committing-workflow hooks, email-workflow scripts, and everything else.




The pre-commit hook is run first, before you even type in a commit message. It’s used to inspect the snapshot that’s about to be committed, to see if you’ve forgotten something, to make sure tests run, or to examine whatever you need to inspect in the code. Exiting non-zero from this hook aborts the commit, although you can bypass it with git commit --no-verify. You can do things like check for code style (run lint or something equivalent), check for trailing whitespace (the default hook does exactly this), or check for appropriate documentation on new methods.

The prepare-commit-msg hook is run before the commit message editor is fired up but after the default message is created. It lets you edit the default message before the commit author sees it. This hook takes a few parameters: the path to the file that holds the commit message so far, the type of commit, and the commit SHA-1 if this is an amended commit. This hook generally isn’t useful for normal commits; rather, it’s good for commits where the default message is auto-generated, such as templated commit messages, merge commits, squashed commits, and amended commits. You may use it in conjunction with a commit template to programmatically insert information.

The commit-msg hook takes one parameter, which again is the path to a temporary file that contains the commit message written by the developer. If this script exits non-zero, Git aborts the commit process, so you can use it to validate your project state or commit message before allowing a commit to go through. In the last section of this chapter, we’ll demonstrate using this hook to check that your commit message is conformant to a required pattern.

After the entire commit process is completed, the post-commit hook runs. It doesn’t take any parameters, but you can easily get the last commit by running git log -1 HEAD. Generally, this script is used for notification or something similar.
Email Workflow Hooks

You can set up three client-side hooks for an email-based workflow. They’re all invoked by the git am command, so if you aren’t using that command in your workflow, you can safely skip to the next section. If you’re taking patches over email prepared by git format-patch, then some of these may be helpful to you.

The first hook that is run is applypatch-msg. It takes a single argument: the name of the temporary file that contains the proposed commit message. Git aborts the patch if this script exits non-zero. You can use this to make sure a commit message is properly formatted, or to normalize the message by having the script edit it in place.

The next hook to run when applying patches via git am is pre-applypatch. Somewhat confusingly, it is run after the patch is applied but before a commit is made, so you can use it to inspect the snapshot before making the commit. You can run tests or otherwise inspect the working tree with this script. If something is missing or the tests don’t pass, exiting non-zero aborts the git am script without committing the patch.

The last hook to run during a git am operation is post-applypatch, which runs after the commit is made. You can use it to notify a group or the author of the patch you pulled in that you’ve done so. You can’t stop the patching process with this script.
Other Client Hooks

The pre-rebase hook runs before you rebase anything and can halt the process by exiting non-zero. You can use this hook to disallow rebasing any commits that have already been pushed. The example pre-rebase hook that Git installs does this, although it makes some assumptions that may not match with your workflow.

The post-rewrite hook is run by commands that replace commits, such as git commit --amend and git rebase (though not by git filter-branch). Its single argument is which command triggered the rewrite, and it receives a list of rewrites on stdin. This hook has many of the same uses as the post-checkout and post-merge hooks.

After you run a successful git checkout, the post-checkout hook runs; you can use it to set up your working directory properly for your project environment. This may mean moving in large binary files that you don’t want source controlled, auto-generating documentation, or something along those lines.

The post-merge hook runs after a successful merge command. You can use it to restore data in the working tree that Git can’t track, such as permissions data. This hook can likewise validate the presence of files external to Git control that you may want copied in when the working tree changes.

The pre-push hook runs during git push, after the remote refs have been updated but before any objects have been transferred. It receives the name and location of the remote as parameters, and a list of to-be-updated refs through stdin. You can use it to validate a set of ref updates before a push occurs (a non-zero exit code will abort the push).

Git occasionally does garbage collection as part of its normal operation, by invoking git gc --auto. The pre-auto-gc hook is invoked just before the garbage collection takes place, and can be used to notify you that this is happening, or to abort the collection if now isn’t a good time.
