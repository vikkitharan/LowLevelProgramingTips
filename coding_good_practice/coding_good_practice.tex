\chapter{Good Coding Practices}


\section{Coding Standard}

\href{https://www.kernel.org/doc/html/v4.10/process/coding-style.html}{Linux kernel coding style} 
\subsection{clang-format}

\subsubsection{Install clang-format}%
\texttt{sudo apt install clang-format-16} installs \texttt{clang-format-16}, it is the latest version avalable as I am wrting. 

\subsubsection{Getting Help}%
\label{ssub:getting_help}
\texttt{clang-format --help}

clangFormat is a set of tools that are built on top of LibFormat. The clangFormat implements automatic source code formatting based on Clang.

clang-format is located in clang/tools/clang-format and can be used to format \texttt{C/C++/Java/JavaScript/JSON/Objective-C/Protobuf/C\#} code.


When the desired code formatting style is different from the available options, the style can be customized using the \texttt{-style="\{key: value, ...\}"} option or by putting your style configuration in the \texttt{.clang-format} or \texttt{\_clang-format} file in your project`s directory and using \texttt{clang-format -style=file}.


\href{https://clang.llvm.org/docs/ClangFormatStyleOptions.html#}{Clang-Format Style Options} 
Clang-Format Style Options describes configurable formatting style options supported by LibFormat and ClangFormat.

An easy way to create the \texttt{.clang-format} file is:
\begin{minted}{bash}
  $ clang-format -style=llvm -dump-config > .clang-format
\end{minted}

\href{https://github.com/torvalds/linux/blob/master/.clang-format}{Linux`s .clang-format file} 

When using -style=file, clang-format for each input file will try to find the .clang-format file located in the closest parent directory of the input file. When the standard input is used, the search is started from the current directory.

When using -style=file:<format\_file\_path>, clang-format for each input file will use the format file located at <format\_file\_path>. The path may be absolute or relative to the working directory.



Disabling Formatting on a Piece of Code

Clang-format understands also special comments that switch formatting in a delimited range. The code between a comment // clang-format off or /* clang-format off */ up to a comment // clang-format on or /* clang-format on */ will not be formatted. The comments themselves will be formatted (aligned) normally.


\subsection{vim integration}
From \href{https://clang.llvm.org/docs/ClangFormat.html}{ClangFormat},

There is an integration for vim which lets you run the clang-format standalone tool on your current buffer, optionally selecting regions to reformat. The integration has the form of a python-file which can be found under \texttt{clang/tools/clang-format/clang-format.py}.

This can be integrated by adding the following to your \texttt{.vimrc}:

\begin{listing}[!ht]
  \begin{minted}{vim}
    map <C-K> :pyf <path-to-this-file>/clang-format.py<cr>
    imap <C-K> <c-o>:pyf <path-to-this-file>/clang-format.py<cr>
  \end{minted}
  \caption{Format selected lindes in vim by clang-format.}
  \label{vim.clang.format}
\end{listing}


The first line enables clang-format for NORMAL and VISUAL mode, the second line adds support for INSERT mode. Change “C-K” to another binding if you need clang-format on a different key (C-K stands for Ctrl+k).

With this integration you can press the bound key and clang-format will format the current line in NORMAL and INSERT mode or the selected region in VISUAL mode. The line or region is extended to the next bigger syntactic entity.

It operates on the current, potentially unsaved buffer and does not create or save any files. To revert a formatting, just undo.

An alternative option is to format changes when saving a file and thus to have a zero-effort integration into the coding workflow. To do this, add this to your .vimrc:


\begin{listing}[!ht]
  \begin{minted}{vim}
    function! Formatonsave()
    let l:formatdiff = 1
    pyf ~/llvm/tools/clang/tools/clang-format/clang-format.py
    endfunction
    autocmd BufWritePre *.h,*.cc,*.cpp call Formatonsave()
  \end{minted}
  \caption{Format a buffer when write.}
  \label{vim.clang.format.all.buffer}
\end{listing}


\subsubsection{Script for patch reformatting}%
\label{ssub:script_for_patch_reformatting}


The python script \texttt{clang/tools/clang-format/clang-format-diff.py} parses the output of a unified diff and reformats all contained lines with clang-format.

Example usage for git users:


\begin{minted}{bash}
  $ git diff -U0 --no-color --relative HEAD^ | clang-format-diff.py -p1 -i
\end{minted}

\subsection{automation - git hooks}



\section{Static Error Check}
\subsection{clang-tidy}
\subsection{vim integration}


\subsection{automation - git hooks}

\section{Documentation}
\subsection{Doxygen}

\subsubsection{Getting help}

\begin{enumerate}
  \item \texttt{doxygen --help} displays doxygen help page.
  \item \href{https://doxygen.nl/manual/index.html}{Doxygen Manual}
\end{enumerate}

To install doxygen run:

\texttt{sudo apt-get install doxygen}

\subsubsection{Configuration file}

To generate a template configuration file, use the following command.

\texttt{ doxygen -g .doxygen\_template.conf}

\subsubsection{Customize the configuration template}

\subsubsection{Layout file}

To generate documentation of source code, use the following command.

\texttt{ doxygen -l .doxygen\_template.conf}

To include the layout update \texttt{LAYOUT\_FILE            = DoxygenLayout.xml} in \texttt{.doxygen\_template.conf} file.

\subsubsection{Customize the layout}
The output layout can be customized. The details are in \href{https://doxygen.nl/manual/customize.html}{Customizing the output}.

\subsubsection{Adding documentation in vim}
I Use \href{https://github.com/vim-scripts/DoxygenToolkit.vim}{DoxygenToolkit} plug in to add doxugen comments.


\section{Reference}

\begin{enumerate}
  \item \href{https://www.youtube.com/watch?v=TtRn3HsOm1s}{YouTube: Doxygen Basics}
\end{enumerate}
